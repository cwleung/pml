\section{Distribution}

\paragraph{Exponential Family}
$p_{F}(x; \theta) = \exp(\langle t(x), \theta\rangle - F(\theta) + k(x))$ \\

\paragraph{Conjugate Prior}
$p(\theta | x) = p_{F}(\theta; \eta) = \exp(\langle t(\theta), \eta\rangle - F(\eta) + k(\theta))$ \\

\paragraph{Posterior}
$p(\theta | x) = p_{F}(\theta; \eta + t(x)) = \exp(\langle t(\theta), \eta + t(x)\rangle - F(\eta + t(x)) + k(\theta))$ \\


\section{KL-Divergence}
\begin{equation}
    \text{KL}(p || q) = \int p(x) \log \frac{p(x)}{q(x)} dx = \mathbb{E}_p \left[ \log \frac{p(x)}{q(x)} \right]
\end{equation}
\subsections{Properties}
\begin{enumerate}
    \item $ \text{KL}(p || q) = \infty \iff \exists x \in \text{supp}(p) \text{ s.t. } q(x) = 0$
\end{enumerate}


\section{Gaussian Distribution}
The following property showing that the actions of Gaussian distribution is also a Gaussian distribution.
\begin{enumerate}
    \item Additive Property
    \begin{equation}
        \mathcal{N}(\mu_1, \Sigma_1) + \mathcal{N}(\mu_2, \Sigma_2) = \mathcal{N}(\mu_1 + \mu_2, \Sigma_1 + \Sigma_2)
    \end{equation}
    \item Multiplicative Property
    \begin{equation}
        \mathcal{N}(\mu_1, \Sigma_1) \times \mathcal{N}(\mu_2, \Sigma_2) = \mathcal{N}(\mu_1 \mu_2, \mu_1^2 \Sigma_2 + \mu_2^2 \Sigma_1 + \Sigma_1 \Sigma_2)
    \end{equation}
    \item Linear Transformation
    \begin{equation}
        \mathcal{N}(\mu, \Sigma) = \mathcal{N}(\mathbf{A} \mathbf{x} + \mathbf{b}, \mathbf{A} \Sigma \mathbf{A}^T)

    \end{equation}
    \item Inverse
    \begin{equation}
        \mathcal{N}(\mu, \Sigma)^{-1} = \mathcal{N}(\Sigma^{-1} \mu, \Sigma^{-1})
    \end{equation}
    \item Determinant
    \begin{equation}
        |\mathcal{N}(\mu, \Sigma)| = \frac{1}{\sqrt{|\Sigma|}} \exp \left( - \frac{1}{2} \mu^T \Sigma^{-1} \mu \right)
    \end{equation}
    \item Trace
    \begin{equation}
        \text{tr}(\mathcal{N}(\mu, \Sigma)) = \mu^T \mu + \text{tr}(\Sigma)
    \end{equation}
    \item Marginalization
    \begin{equation}
        \mathcal{N}(\mu, \Sigma) = \mathcal{N}(\mu_1, \Sigma_{11}) \times \mathcal{N}(\mu_2, \Sigma_{22})
    \end{equation}
    \item Conditioning
    \begin{equation}
        \mathcal{N}(\mu, \Sigma) = \mathcal{N}(\mu_1, \Sigma_{11}) \times \mathcal{N}(\mu_2, \Sigma_{22})
    \end{equation}
\end{enumerate}